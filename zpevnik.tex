% !TEX TS-program = pdflatex
% !TEX encoding = UTF-8 Unicode
\documentclass[15pt]{article} 

%%% ENCODING
\usepackage[utf8]{inputenc}
\usepackage[czech]{babel}
\usepackage[T1]{fontenc}

%%% FONTS
\usepackage{lmodern} 
\usepackage[scaled]{helvet} 

%%% PAGE DIMENSIONS
\usepackage[top=25pt, bottom=33pt, left=30pt, right=30pt]{geometry}
\geometry{a5paper}
\pagestyle{empty} % no header nor footer
\pagenumbering{gobble} % suppress page numbering on first page

% MISCELLANEOUS
\usepackage{graphicx} % support the \includegraphics command and options
\usepackage{setspace} % spacing between lines
\usepackage{microtype} % micro-typographic font refinement
\usepackage{paralist} % very flexible & customisable lists (eg. enumerate/itemize, etc.)
\usepackage{subfig} % make it possible to include more than one captioned figure/table in a single float
\usepackage{truncate} %for truncating song titles in ToC
\usepackage{xfrac} % better fractions, E.G. 3/4 etc.
\usepackage[parfill]{parskip} % Activate to begin paragraphs with an empty line rather than an indent
\usepackage{needspace}

% CHORDS & MUSIC
\usepackage{gtrcrd} % chords above lyrics
\usepackage{gchords} % chord diagrams
\usepackage{musixtex} % typeset music notation

% TABLE OF CONTENTS
\usepackage{tableof} % for tagged table of contents
\usepackage{etoc} % for customized typesetting of tocs
% introduce sectioning element `song'
\etocsetlevel{song}{1}
% define ToC line style for song
% \etocsetstyle{<levelname>}{<start>}{<prefix>}{<contents>}{<finish>}
\etocsetstyle{song}{} 
{\leavevmode\leftskip 0pt\relax}
{\hskip 4pt{\etocname}\nobreak\hfill\nobreak\par}{}

% CUSTOM MACROS
\input{macros.tex}
\input{chords_adjustments.tex}


\begin{document}

%% ToC categorized
\categoryHeader{KATEGORIE}
\begin{multicols}{2}
  \setlength{\columnsep}{20pt}
  \raggedcolumns
  \categoryHeader{Zahraniční}
    \categoryToc{Anglické}{anglicke}
    \categoryToc{Slovenské}{slovenske}
  
  \categoryHeader{Písničkáři}
    \categoryToc{Tomáš Klus}{klus}
    \categoryToc{Karel Kryl}{kryl}
    \categoryToc{Jaromír Nohavica}{nohavica}
    \categoryToc{Karel Plíhal}{plihal}
    
  \categoryHeader{České kapely}
    \categoryToc{Arakain, Brichta, BSP}{arakain, brichta, bsp}
    \categoryToc{Buty}{buty}
    \categoryToc{Chinaski}{chinaski}
    \categoryToc{Kabát}{kabat}
    \categoryToc{Lucie}{lucie}
    \categoryToc{Olympic}{olympic}
    \categoryToc{Žlutý pes}{zlutypes}
    \categoryToc{ostatní}{ceskekapely}

  \categoryHeader{Folk, country}
    \categoryToc{Spirituál kvintet, Nedvědi, Brontosauři}{spiritual, nedvedi, brontosauri}
    \categoryToc{Čechomor}{cechomor}
    \categoryToc{Robert Křesťan}{krestan}
    \categoryToc{Pavel Lohonka Žalman}{zalman}
    \categoryToc{Nerez, Zuzana Navarová}{nerez, navarova}
    \categoryToc{ostatní}{folk, raduza}

  \categoryHeader{Čeští skladatelé}
    \categoryToc{P. Hapka + M. Horáček}{hapka}
    \categoryToc{J. Voskovec + J. Werich}{vw}
    \categoryToc{Z. Svěrák + J. Uhlíř}{sverakuhlir, uhlirsverak}
    \categoryToc{Ivan Mládek}{mladek}
    \categoryToc{J. Suchý + J. Šlitr}{suchyslitr}

  \categoryHeader{Filmové, muzikálové}
    \categoryToc{Ivan Hlas}{hlas}
    \categoryToc{ostatní}{filmove}
\end{multicols}
\clearpage

%% full list of songs
\etocruledstyle{\textbf{\large\ \uppercase{Obsah}\ }}
\nexttocwithtags{}{}\tableofcontents
\clearpage

%% document settings
\Large % do not remove! Otherwise titles will get messed up -_-
\fontsize{14pt}{2.9ex} \selectfont
\setlength{\parskip}{2ex} % vertical space between paragraphs
\crdheight=2.5ex

% typesetting chords on their own will not print them;
% they need to be in the printchords macro
% \DoNotPrintChordsByDefault

%% --------------  SONGS --------------------

\song{Faník}{Anděl, Karel Kryl feat. Janča}{30pt}{1}{
\printchords{

\verse{1}{}\G{}My někte\Em{}rýho objektu,\G{} po konci \D[7]{}jedné akce,}\\
\G{}vzpomínku \Em{}jsme si odnesli \G{}na látek\D[7]{} interakce.\\
\G{}Jak a kdy \Em{}vznikaj krystaly, \G{}nám Faník \D[7]{}odpřednášel,\\
\G{}takže \Em{}na příštím ŠMFku \G{}hlas účastníků \D[7]{} zazněl. 
\printchords{

\chorus{}\G{}Faníku, \Em{}prosím, věř nám,\G{} přišli jsme \D[7]{}žádat,\\
\G{}přednášku \Em{}ještě jednu\G{} nechceš \D[7]{}pořádat?\\
\G{}o Maxwellkách, \Em{} \hskip 2.3em \D[7]{}a krysta\G{}lech,\\
\G{}o o Maxwellkách, \Em{} \hskip 2.3em \D[7]{}a krysta\G{}lech.
}

\verse{2}{} Pak neměli jsme vůbec spát, přednáška od Faníka\\
však po třech nočkách za sebou je chyba převeliká.\\
My spink v tu ránu hodili, Faníka jsme zklamali,\\
Celou půldruhou hodinu jsme o krystalech snili.\\
\textbf{R:} 

\verse{3}{}My jsme to ale nevzdali, za Fáňou šli jsme znova,\\
Že tenhle blok přednáškový nám síla kofolová \\
čarovnou mocí zažene náš spánku nedostatek \\
a on že na nás nepozná, že včera byl výsadek.\\
\textbf{R: }
 }



% \toftagthis{folk}
% \song{Bláznova ukolébavka}{Pavel Dydovič}{40pt}{1}{
% \printchords{%
% \verse{1}{}\D{}Máš, má ovečko, \A{}dávno spát, i \G{}píseň ptáků \D{}končí.\\
% }
% \D{}Kvůli nám přestal \A{}vítr vát, jen \G{}můra zírá \D{}zvenčí.\\
% \printchords{%
% Já \A{}znám její zášť, tak \G{}vyhledej skrýš,\\
% zas \A{}má bílej plášť a \G{}v okně je \A{}mříž. 
% }

% \printchords{%
% \chorus{}\D{}Máš, má ovečko, \A{}dávno spát\\
% a \G{}můžeš hřát, ty mně \E{}můžeš hřát,\\
% vždyť \D{}přijdou se \G{}ptát,\\
% zítra zas \D{}přijdou se \G{}ptát,\\
% jestli ty v \D{}mých předsta\G{}vách už \D{}mizíš. 
% }

% \verse{2}{}\D{}Máš, má ovečko, \A{}dávno spát,\\ 
% dnes \G{}máme půlnoc \D{}temnou.\\
% \D{}Ráno budou nám \A{}bláznů lát,\\
% že \G{}ráda snídáš \D{}se mnou.\\
% Proč \A{}měl bych jim lhát, že \G{}jsem tady sám,\\
% když \A{}tebe mám rád, když \G{}tebe tu \A{}mám.\\
% \textbf{R:}
% }




\toftagthis{brichta}
\song{\textls[-10]{Fáňa s gelíkem ve vlasech}}{Dívka s perlami ve vlasech, Aleš Brichta feat. Tonda}{60pt}{0.9}{
\setlength{\parskip}{1ex} % vertical space between paragraphs
\verse{1}{}\Em{}Tak nás tu \D{}máš, ně\Am{}jak se \Em{}mračíš,\\
Vybledlej smích, tu na baru.\\
S nádhernym knírkem, účes svůj strácíš,\\
Fantasie naše plníš pomalu.

\chorusAlt{R1:}{-29pt}No tak Fá\G{}ňo, co \D{}chceš mi říct,\\
\Am{}máš už mašli, mož\Em{}ná i víc,\\
Fáňo, na co se ptáš, vlnku na čele máš,\\
Nevím Fáňo, co tam dát dál, nechtěl jsi nic, já ti to dal,\\
Fáňo, na co se ptáš, gelík ve vlasech máš.

\verse{2}{}Gelíku glued víc radši ne,\\
Máš ho tam moc, skoro teče...\\
Ouško snad drží, i když moc ne,\\
Lepí  jen málo,  klid,  člověče.

\chorusAlt{R2:}{-30pt}No tak Fáňo, kdo ti to vzal,\\
Kdo ti sebral cudnost i respekt,\\
Fáňo, na co je pláč, kočkoklukem být máš.\\
Vždyť už Fáňo gelíček máš, \\
Tak proč protest, však je prd platný.\\
Fáňo, na co je vzpor, je z tebe pěkný tvor.

\verse{3}{}Chtěli jsme víc, pro naše touhy,\\
My, účastníci, chceme co chcem...\\
Co vlastně zbývá? Švarcnegr pouhý... \\
S troškou kofoly a sójovkou

\textbf{R2: + R1:}
}


\toftagthis{filmove}
\song{Faníku, ty slůně}{Lásko má, já stůňu; Karel Svoboda feat. FK (fanklub}{25pt}{1}{
\capo{4}

\verse{1}{}\Am{}Já, ač mám spánek \Em{}bezesný, mně \D{}včera \E{}sen se \Am{}zdál.\\
\Am{}I když teď už \Em{}není org, je \D{}stále\E{} milo\F{}ván, \G{}řekli:

\chorus{}\uv{\C{}Faníku, ty \G{}slůně, pro nás \Dm{}budeš vždycky \Am{}král,\\
kvůli \C[7]{}nám se vzdáváš \F{}spánku, a scé\C{}nek kt\G{}eré jsi\E{}hrál.}

\verse{2}{}Ač sous nebývá poklidný, Dnes nevím, kudy kam.
Trápí nás mlsník prázdný, A k tomu Faník sám... řekli

\chorus{}\uv{Faníku ty,\ldots{}\\
\hskip 1em \ldots{}scének, které jsi\C{} hrál.} \G{}Řekli:

\chorus{}\uv{Faníku ty,\ldots{}\\
\hskip 1em \ldots\ scének, které jsi \A{}hrál.}\\
}



\toftagthis{ceskekapely}
\song{No tak, Fáňo}{Jelen, Magdaléna feat. Lukáš}{40pt}{1}{
\verse{1}\A{}Arktický krokodýl jsi,\D[add9]{} a ruce lovíš,\\
\A{}když jezdíš na kapotě.\D[add9]{}.\\
Arktický krokodýl jsi, a ruce lovíš,\\
hlavně tu Johna Donta, té se snad najíš.\\
Orguj \A{}dál, prosím tě, \D[add9]{}nenech se prosit se \A{}dál,\\
\D[add9]{}nenech se! 

\verse{2}Sedíme na přednášce, přichází ráno,\\
ty mluvíš o krystalech.\\
Sedíme na přednášce, přichází ráno\\
a mně se klíží očí.\\
Orguj dál, prosím tě, nenech se prosit se dál,\\
nenech se!

\chorus{}\Fsm{}Hodiny se zastaví a \E{}během tvojí přednášky,\\
\A{}No tak Fáňo, \A{}tvoje vousy\\
stále jenom orgují a vysvětljí Maxvelky,\\
oholíš je, pročpak asi.\\
Hodiny se zastaví a během tvojí přednášky\\
jdem\A{} spát.\D[add9]{}
\clearpage
\verse{3}Stojíme na parket a hudba už zní,\\
Učíš nás tancování,\\
Stojíme na parket a hudba už zní,\\
Nauč nás všechny kroky.\\
Orguj dál, prosím tě, nenech se prosit se dál,\\
nenech se!\\
\textbf{R:}

\chorus{}Hodiny se zastaví a během tvojí přednášky,\\
No tak Fáňo, tvoje vousy\\
stále jenom orgují a vysvětljí Maxvelky,\\
oholíš je, pročpak asi.\\
Hodiny se zastaví a během tvojí přednášky \\
máš těžký srdce, nagelovaný vlasy,\\
všichni účastníci spí a tobě to snad nevadí,\\
nenech se!\\
}



\toftagthis{ceskekapely}
\song{Fáňo}{Sáro, Traband feat. Tonda}{30pt}{1}{
\chorus{}\Am{}Fáňo, \Em{}Fáňo, \F{}tak se zase\C{}stalo,\\
že \F{}účastníci \C{}všichni, tvou \F{}přednášku pro\G{}spali.\\
Fáňo, Fáňo, jak moc a nebo málo\\
Ti chybí proto aby pozor zase dávali...

\verse{1}Vzor struktur krystalů určují pomalu,\\
A u všelijakých tvarů vidí čtverec jen...\\
Z odehraných her se týmy zpátky vrací, \\
jejich smích burácí, a chtějí více scén!\\
\textbf{R:}

\verse{2}Mileta v zámku balí saky paky,\\
Než přivedou jí draky, pak mává na pozdrav.\\
A její Lord Friendzone má v každé ruce meče,\\
a krev krále už teče, nastupuje Jeff\\
\textbf{R:}

\verse{3}Huli s hulinenem, královští stráže, \\
Dělaj jak pan káže, nikdo nejde za králem!\\
Tak u brány stojí, jedno kopí mají, \\
a vrazi se skrývají, jedou kočárem...\\
\textbf{R:}

\verse{4}John Don't běhá za zdí, snaží se co může, \\
moc avšak nezmůže, neví totiž nic.\\
Propadá se ledem, fightí krokodýly, \\
přichází o díly ač se snaží sebevíc...\\

\chorus{}Fáňo, Fáňo, pomalu a líně,\\
S notýskem ve klíně chci přednášku prosedět.\\
Fáňo Fáňo Fáňo Fáňo už je skoro ráno...\\
Tak vstávej a ukaž nám krystalický svět!\\

\F{}Fáňo, \C{}Fáňo, \F{}vstávej, milý \C{}Fáňo,\\
\F{}O Maxvellkách \Dm{}pojď nám předná\C[maj]{}šet.\\
}




\song{Stín objektů}{Stín Katedrál, Karel Svoboda feat. Janča}{40pt}{1}{
\verse{1}\A{}Stín \D{}objektů, \A{}půl orgů s Faní \G{}kem, \D{}jé \E{}jé,\\
\A{}svůj \D{}ideál, \A{}sen, co ve \H[7]{}vlau \E{}sním.\E[7]{}\\
Z kofoly čaj, ve vlastním hrnečku, jéjé,\\
Co ti dál mám, řekni, dárkem dát?\\
(Máslo)


\chorus{}\C{}Vem si, co \D{}chceš, mlsník \G{}ro--zebe\D{}reš,\\
mlíka \C{}půl, cukr, \D{}sůl, kilo--\G{}dvě. \G{}\\
\C{}Ber, tady \D{}máš pět \G{}kilo ci\D{}bule,\\
Z ku\C{}chyně, jenom \D{}ber se Fáňou \E{}též, \Hm[7]{}jé \hskip 1.8em \E[7]{}jé.

\verse{2}\A{}Náš \D{}ideál, \A{}víš, co účastník \G{}rád, \D{}jé \E{}jé,\\
\A{}stín \D{}objektů, \A{}sen, co \H[7]{}ve vlaku \D{}s\E[7]{}ním \A{}.\\
\textbf{R:}

\verse{3} Náš ideál, víš \dots

\revrpt{} \H[7]{}Ten, co ve \E{}vlaku \A{}sním. \rpt{}\\
}


\toftagthis{nohavica}
\song{Když jsem se stal účastníkem}{Když mě brali za vojáka, Jaromír Nohavica feat. Tonda}{20pt}{0.86}{
\verse{1}{}\Am{}Když jsem se stal účast\C{}níkem, \G{}narvali mě do vla\C{}ku,\\
\Dm{}ŠMFko je jako \Am{}seskok, \E{}jenomže bez padá\F{}ku,\\
-\G{}la, -\C{}la, -\G{}la, \Am{}jenomže bez \E{}padá\Am{}ku.

\verse{2}{}Zavřeli mě do objektu, začali mě učiti, \\
Jak spánkový režim ztratit, kofolu že máš píti,\\
-ti, -ti, -ti kofolu že máš píti.

\verse{3}{}Na pokoji v deset večer, ke zdi jsem se přitulil,\\
Třetí den v kuse zas noční, skoro jsem si zabulil,\\
-il, -il, -il skoro jsem si zabulil

\verse{4}{}Když jsem se pak ráno vzbudil, únavou jsem umíral,\\
Mlsník naštestí byl plný, tak kofolu jsem si dal, \\
Dal, dal, dal tak kofolu jsem si dal.

\verse{5}{}Orgové furt jenom chrápou, i když za to nemůžou, \\
Oni ten spánek fakt potřebují, když účastníci je zmůžou,\\
-ou, -ou, -ou účastníci je zmůžou

\verse{6}{} Faník má fakt pěkný vlasy, na baru ho potkali,\\
Doběhli na pokoj proooo gel, účes mu předělali, \\
-li, -li, -li účes mu předělali

\verse{7}{} Co je komu do ŠMFka, když ho sám nezažije, 
na shledanou, budete mi chybět, papa, čau a adieu.
-jé, -jé, -jé ono krásně se žije, 
-jé, -jé, -jé než kofola se dopije....

}




%\song{Stín katedrál}{Karel Svoboda}{40pt}{1}{
%\verse{1}\A{}Stín \D{}katedrál, \A{}půl nebe s bůhví \G{}čím, \D{}jé \E{}jé,\\
%\A{}svůj \D{}ideál, \A{}sen, co si \H[7]{}dávám \E{}zdát.\E[7]{}\\
%Z úsměvů šál, dům nebo básní rým, jéjé\\
%jé, co ti dál, mám, řekni, dárkem dát.

%\chorus{}\C{}Přej si, co \D{}chceš, zlatý \G{}důl nebo \D{}věž,\\
%sladkou \C{}sůl, smutný \D{}ráj, suchý \G{}déšť. \G{}\\
%\C{}Ber, tady \D{}máš mořskou \G{}pláň nebo \D{}pláž,\\
%hudbu \C{}sfér, jenom \D{}ber se mnou \E{}též, \Hm[7]{}jé \hskip 1.8em \E[7]{}jé.

%\verse{2}\A{}Můj \D{}ideál, \A{}víš, to co já mám \G{}rád, \D{}jé \E{}jé,\\
%\A{}stín \D{}katedrál, \A{}sen, co si \H[7]{}k ránu \D{}dá - \E[7]{}vám \A{}zdát.\\
%\textbf{R:}

%\verse{3} Můj ideál, víš \dots

%\revrpt{} \H[7]{}Sen, co se \E{}nám bude \A{}zdát. \rpt{}\\
%}

% \toftagthis{anglicke}
% \song{Suicide is Painless}{M*A*S*H theme}{60pt}{1.05}{
% \vskip 8pt
% \verse{1}Through \Dm[7]{}early morning \G[7]{}fog I see\\
% \C[maj]{}visions of the \Am[7]{}things to be,\\
% the \Dm[7]{}pains that are with\G[7]{}held for me,\\
% I \C{}realize and \Am[7]{}I can see\A[7]{}.

% \chorus{}That \Dm[7]{}suicide is \G[7]{}painless,\\
% it \C[maj]{}brings on many \Am[7]{}changes\\
% and \F[maj]{}I can \Am{}take or \Dm[7]{}leave it \G[7]{}if I \Am{}please.

% \verse{2}Try to find a way to make\\
% all our little joys relate\\
% without that ever present hate,\\
% but now I know that it's too late.\\
% And\ldots{}+ \textbf{R:}


% <DO NOT REMOVE!> 
%% This hack addresses the bug of tableof package (I think, before it was ok):
%% if \clearpage is right before end{document}, it breaks (wont even typeset)
~\\
% </DO NOT REMOVE> 


% end of multicols in toc, needs to be after last title
\addtocontents{lem}{\protect\end{multicols}} 

\end{document}
